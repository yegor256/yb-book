% (The MIT License)
%
% Copyright (c) 2021 Yegor Bugayenko
%
% Permission is hereby granted, free of charge, to any person obtaining a copy
% of this software and associated documentation files (the 'Software'), to deal
% in the Software without restriction, including without limitation the rights
% to use, copy, modify, merge, publish, distribute, sublicense, and/or sell
% copies of the Software, and to permit persons to whom the Software is
% furnished to do so, subject to the following conditions:
%
% The above copyright notice and this permission notice shall be included in all
% copies or substantial portions of the Software.
%
% THE SOFTWARE IS PROVIDED 'AS IS', WITHOUT WARRANTY OF ANY KIND, EXPRESS OR
% IMPLIED, INCLUDING BUT NOT LIMITED TO THE WARRANTIES OF MERCHANTABILITY,
% FITNESS FOR A PARTICULAR PURPOSE AND NONINFRINGEMENT. IN NO EVENT SHALL THE
% AUTHORS OR COPYRIGHT HOLDERS BE LIABLE FOR ANY CLAIM, DAMAGES OR OTHER
% LIABILITY, WHETHER IN AN ACTION OF CONTRACT, TORT OR OTHERWISE, ARISING FROM,
% OUT OF OR IN CONNECTION WITH THE SOFTWARE OR THE USE OR OTHER DEALINGS IN THE
% SOFTWARE.

\documentclass[compact,manuscript]{./ybook}
\renewcommand*\theversion{0.0.0}
\renewcommand*\thedate{00.00.0000}
\renewcommand*\thetitle{\LaTeX{} Class \ff{ybook}}
\renewcommand*\theauthor{\nospell{Yegor Bugayenko}}
\begin{document}

\maketitle

\ybPrintTOC

\chapter{Overview}

\section{Purpose}

\index{Amazon}
The provided class \ff{ybook} helps me design my books and
publish them on Amazon. You are welcome to use it too. It's easy:

\begin{listing}
\documentclass{ybook}
\begin{document}
Hello, world!
\end{document}
\end{listing}

There are a few class options you can use:

\begin{itemize}
    \item \ff{compact} --- when you need to make text more compact
    and take less vertical space;

    \item \ff{manuscript} --- when the format is not for Amazon printing,
    but for some other purposes (the page size is A4);

    \item \ff{draft} --- when it's a draft for reviewers (the page size is A4);

    \item \ff{cover=book.pdf} --- when you print a cover and the book is in the
    \ff{book.pdf} file.
\end{itemize}

\section{Cover Page}

A cover page must have a special formatting. You do it this way
(provided your book already exists in `book.pdf`):

\begin{listing}
\documentclass[cover=book.pdf]{ybook}
\renewcommand*\theauthor{Jeff Lebowski}
\renewcommand*\theversion{0.1}
\renewcommand*\theyear{2021}
\renewcommand*\thevolume{1}
\renewcommand*\theprice{\$14.99}
\renewcommand*\thespinetext{new book, vol.1}
\renewcommand*\thetitle{A New Book}
\renewcommand*\theimage{images/cactus}
\renewcommand*\thetldr{Put your TL;DR text here...}
\begin{document}
\ybPrintCover{}
\end{document}\end{listing}

The PDF file generated most likely will be accepted by Amazon without
any questions.

\section{Commands}

There are a few new commands you may want to learn and use:

\index{font}
\ff{\textbackslash{}ff} makes your code fixed-font, like \ff{this one}.

\ff{\textbackslash{}tbd} makes some text \tbd{highlighted}, in order to indicate
that it will be written later.

You are welcome to suggest additional commands, but the style
of my books is intentionally as simple as possible, avoiding formatting
as much as possible.

\ybPrintIndex

\end{document}